%------------------------------------
% Dario Taraborelli
% Typesetting your academic CV in LaTeX
%
% URL: http://nitens.org/taraborelli/cvtex
% DISCLAIMER: This template is provided for free and without any guarantee 
% that it will correctly compile on your system if you have a non-standard  
% configuration.
% Some rights reserved: http://creativecommons.org/licenses/by-sa/3.0/
%------------------------------------

%!TEX TS-program = xelatex
%!TEX encoding = UTF-8 Unicode

\documentclass[11pt, a4paper]{article}
\usepackage{fontspec} 
\usepackage{enumitem}
\usepackage{multicol}
\setlist{nosep}
% DOCUMENT LAYOUT
\usepackage{geometry} 
\geometry{a4paper, textwidth=6.75in, textheight=8.5in, marginparsep=1pt, marginparwidth=.6in, top=2cm, bottom=1cm}
\setlength\parindent{0in}
\pagenumbering{gobble}
% FONTS
\usepackage{xunicode}
\usepackage{xltxtra}
\defaultfontfeatures{Mapping=tex-text} % converts LaTeX specials (``quotes'' --- dashes etc.) to unicode
\setmonofont[Scale=0.8]{Monaco} 
\setsansfont[Scale=0.9]{Optima Regular} 
% ---- CUSTOM AMPERSAND
\newcommand{\amper}{{\fontspec[Scale=.90]{Hoefler Text}\selectfont\itshape\&}}
\renewcommand{\familydefault}{\sfdefault}
% HEADINGS
\usepackage{sectsty} 
\usepackage[normalem]{ulem} 
\sectionfont{\sffamily\mdseries\LARGE} 
\subsectionfont{\sffamily\mdseries\normalsize} 
\subsubsectionfont{\sffamily\bfseries\normalsize} 
\usepackage[compact]{titlesec}
\titlespacing{\section}{0cm}{2mm}{.1pc}
\titlespacing{\subsection}{0cm}{1mm}{*0}
\titlespacing{\subsubsection}{1mm}{*0}{*0}
% PDF SETUP
% ---- FILL IN HERE THE DOC TITLE AND AUTHOR
\usepackage[xetex, bookmarks, colorlinks, breaklinks, pdftitle={Julius Park Resume},pdfauthor={Julius Park }]{hyperref}  
\hypersetup{linkcolor=blue,citecolor=blue,filecolor=black,urlcolor=blue} 

% DOCUMENT
\begin{document}
\textsf{\Huge Julius Park}\hspace{4mm}484-995-1301 | \href{mailto:juliusgeo@gmail.com}{juliusgeo@gmail.com} | \href{github.com/juliusgeo}{github.com/juliusgeo}
\section*{Education}
\noindent Candidate for \textsc{BSc} in Computer Engineering, Drexel University, \textsc{GPA: 3.64,} \textit{2016-2021}

\vspace{-1mm}
\section*{Skills}
\textbf{Languages} Python, C\#, C++, Julia, Swift, MATLAB, VHDL\\ 
\textbf{Technologies} Git, Perforce, Visual Studio, Tensorflow, NumPy, Pandas, bash, zsh, Jira, Docker, .NET, QGIS
\section*{Experience}
\textbf{\emph{Software Engineering Intern}, MongoDB, San Francisco}, \textit{June 2020--Aug. 2020}
\begin{itemize}
	\item Led development of a standalone Python package for debugging and profiling code that utilizes MongoDB's Python driver
	\item Employed monkey-patching to develop command line interface for debugging existing Python scripts
	\item Integrated multiple testing methods into software package including unit testing and specification testing
	\item Contributed multiple features and bug fixes to official MongoDB Python driver which has over 6 million downloads a month, in addition to the official NumPy integration package
     \end{itemize}
\vspace{1mm}
\textbf{\emph{Software Engineer}, Analytical Graphics Inc., Exton}, \textit{Apr. 2019--Sep. 2019}
\begin{itemize}
	\item Worked on Scalability team primarily developing Software Tool Kit, AGI's flagship aerospace simulation software, focusing on parallel processing and supporting macro scripting engines
     \item Led migration of STK's ~3mil LOC code base to comply with C++20 standard and remove deprecated code, including validation of pre-processor tags to ensure deprecated code is not added in the future
     \item Changed third-party libraries and their build scripts to remove dependence on boost::filesystem as part of effort to switch over to std::filesystem
     \item Developed testing framework for various parallel processing features including binary save and load and parallel volumetric calculation
     \item Led creation of plugin to graphically visualize dependencies between STK objects using MSAGL
     \end{itemize}
\vspace{1mm}
\textbf{\emph{Software Engineer}, Drexel University CCI, Philadelphia}, \textit{Apr. 2018--Mar. 2019}
\begin{itemize}
     \item Led development of application using C\# to interface with a medical device designed for physical therapists
     \item Designed data models, coded vast majority (99\% of commits) from start of project to release
     \item Built interface including live visualization of data from medical device, and export of data in compliance with HIPAA
     \item Used multithreading to enhance performance and allow real-time plotting of data over Bluetooth serial connection
     \end{itemize}
\vspace{1mm}
\textbf{\emph{Software Engineer}, Drexel University, Philadelphia}, \textit{Feb. 2017--Apr. 2018} 
\begin{itemize}
 \item Utilized Python and NumPy, Python Imaging Library, Cython, and OpenCV to create custom algorithm to quantify architectural changes in images
     \item Wrote image comparison software for research project NSF \#1562515
     \item Collaborated with faculty at both Drexel and UT Austin
     \end{itemize}
\vspace{0mm}

\section*{Extracurricular \& Personal Projects}
\textbf{OpenStreetMapPlotter.jl}
\begin{itemize} 
     \item Developed a package for Julia to parse, display, and export maps in OpenStreetMap format
     \item Integrated with Overpass API to allow easy access to map data and developed MapCSS parser to allow custom theming of resulting map plots from CSS files
     \item Package was accepted into official Julia package manager
     \end{itemize}
          \vspace{1mm}
\textbf{BikingElevationMap}
\begin{itemize} 
     \item Developed Python code using GeoPandas to generate custom elevation maps from OpenStreetMap data and export resulting geometry to .shp or .geojson format
     \item Used QGIS to style resulting files and create finished maps
     \end{itemize}
     \vspace{1mm}
\textbf{DART Robotics Team} \textit{2018}
\begin{itemize} 
     \item Lead for image processing on MATE aquatic robotics team
     \item Using OpenCV to process images from aquatic robot, including object detection and OCR
     \end{itemize}
 \vspace{1mm}
  
\end{document}
